%% LyX 2.1.3 created this file.  For more info, see http://www.lyx.org/.
%% Do not edit unless you really know what you are doing.
\documentclass[10pt]{beamer}
\usepackage{etex}

\usepackage[T1]{fontenc}
\usepackage{textpos} 
\usepackage{hyperref}
\usepackage{amsmath,amsthm,amsfonts,nicefrac,mathabx,amssymb}
 \usepackage[subnum]{cases}
\usepackage{calligra, mathrsfs}
%\usepackage{natbib}
\usepackage{booktabs}
%\bibpunct{(}{)}{;}{a}{,}{,}
\usepackage[english]{babel}
\usepackage[latin1]{inputenc}
\usepackage{helvet}
\usepackage{graphicx}
\usepackage{color}
\usepackage{multirow,dcolumn}
\usepackage{ragged2e}
\usepackage{xcolor}
\usepackage{colortbl}
\usepackage{booktabs}
\usepackage{enumitem}
\usepackage{url}
\usepackage{bibentry}
\usepackage{chngcntr}

\ifx\hypersetup\undefined
  \AtBeginDocument{%
    \hypersetup{unicode=true,pdfusetitle,
 bookmarks=true,bookmarksnumbered=false,bookmarksopen=false,
 breaklinks=false,pdfborder={0 0 0},backref=false,colorlinks=false}
  }
\else
  \hypersetup{unicode=true,pdfusetitle,
 bookmarks=true,bookmarksnumbered=false,bookmarksopen=false,
 breaklinks=false,pdfborder={0 0 0},backref=false,colorlinks=false}
\fi

\colorlet{tablesubheadcolor}{gray!25}
\colorlet{tableheadcolor}{gray!40}
\colorlet{tablerowcolor}{gray!15.0}
\usetheme{Gesis}
%\setbeamertemplate{navigation symbols}{}
\setbeamertemplate{footline}[frame number]%{\hspace*{.2cm}\insertframenumber}
\setbeamerfont{caption}{size=\footnotesize}
\usefonttheme[onlylarge]{structuresmallcapsserif} % alte Schrift


\definecolor{hellgrau}{rgb}   {0.109375,  0.40625,   0.51953125}
\definecolor{dunkelgrau}{rgb} {0.009375,  0.30625,   0.41953125}
\definecolor{dunkelgrau2}{rgb}{0.009375,  0.20625,   0.31953125}
\definecolor{hellbraun}{rgb}  {0.9140625, 0.8984375, 0.8046875}
\definecolor{hellbraun2}{rgb} {.95,       0.9,       0.8}
\definecolor{alertred}{rgb}   {0.8515625, 0.3828125, 0.08984375}
\definecolor{orange}{rgb}{1,0.5,0}


\setbeamercolor{firstsecslide}{fg=white,bg=dunkelgrau}
\setbeamertemplate{blocks}[rounded][shadow=true]

\newcolumntype{d}[1]{D{.}{.}{#1}}

\newcommand{\emphred}[1]{\textcolor{alertred}{#1}}
\newcommand{\emphcol}[1]{\textcolor{dunkelgrau}{\slshape #1}}

\newcommand{\eqname}[1]{\tag*{#1}} %equation title

\newenvironment{frcseries}{\fontfamily{frc}\selectfont}{}
\newcommand{\textfrc}[1]{{\frcseries#1}}
\newcommand{\mathfrc}[1]{\text{\textfrc{#1}}}


\setcounter{tocdepth}{1}
\setbeamercolor*{section in toc}{fg=hellgrau}
\setbeamertemplate{bibliography item}[default]


\setcounter{secnumdepth}{3}
\setcounter{tocdepth}{3}


\newcommand{\E}[1]{\text{E}\left(#1\right)}
\newcommand{\V}[1]{\text{V}\left(#1\right)}
\newcommand{\Vest}[1]{\widehat{\text{V}}\left(#1\right)}
\newcommand{\MSE}[1]{\text{MSE}\left(#1\right)}
\newcommand{\COV}[2]{\text{COV}\left(#1,\,#2\right)}
\newcommand{\deff}{\ensuremath{\text{\slshape deff}}}
\newcommand{\deffhat}{\ensuremath{\widehat{\text{\slshape deff}}}}
\newcommand{\deffc}{\ensuremath{\text{\slshape deff}_{\text{\slshape c}}}}
\newcommand{\deffhatc}{\ensuremath{\widehat{\text{\slshape deff}}_{\text{\slshape c}}}}
\newcommand{\deffp}{\ensuremath{\text{\slshape deff}_{\text{\slshape p}}}}
\newcommand{\neff}{\ensuremath{n_\text{\slshape eff}}}
\newcommand{\nnet}{\ensuremath{n_\text{\slshape net}}}
\newcommand{\ngross}{\ensuremath{n_\text{\slshape gross}}}
%roman numbers in equation
\newcommand{\RN}[1]{%
  \textup{\uppercase\expandafter{\romannumeral#1}}%
}


\makeatletter

\addtobeamertemplate{frametitle}{}{%
\begin{textblock*}{100mm}(.91\textwidth,-1cm)
\includegraphics[height=1cm,width=2cm]{graphs/logos/GESIS_Logo_kompakt_en.jpg}
\end{textblock*}}



%%%%%%%%%%%%%%%%%%%%%%%%%%%%%% LyX specific LaTeX commands.
\providecommand{\LyX}{\texorpdfstring%
  {L\kern-.1667em\lower.25em\hbox{Y}\kern-.125emX\@}
  {LyX}}

%%%%%%%%%%%%%%%%%%%%%%%%%%%%% Textclass specific LaTeX commands.
% this default might be overridden by plain title style
 \newcommand\makebeamertitle{\frame{\maketitle}}%
 % (ERT) argument for the TOC
 \AtBeginDocument{%
   \let\origtableofcontents=\tableofcontents
   \def\tableofcontents{\@ifnextchar[{\origtableofcontents}{\gobbletableofcontents}}
   \def\gobbletableofcontents#1{\origtableofcontents}
 }

%%%%%%%%%%%%%%%%%%%%%%%%%%%%%% User specified LaTeX commands.
%\usetheme{Montpellier}%\usetheme{PaloAlto}

\makeatother

\usepackage{Sweave}
\begin{document}
\Sconcordance{concordance:GESISsusc2015_Day2.tex:GESISsusc2015_Day2.Rnw:%
1 133 1 49 0 1 4 86 1 3 0 29 1 1 11 124 1 1 44 131 1 11 0 114 1 1 %
101 79 1}

\begin{Schunk}
\begin{Sinput}
> library(knitr)
> opts_chunk$set(fig.path='graphs/beamer-',fig.align='center',fig.show='hold',size='footnotesize')
\end{Sinput}
\end{Schunk}


\title[Complex Sampling Designs]{Sampling and Estimation}   
\subtitle{Day 2: Complex Sampling Designs}

\author{Stefan Zins\thanks{\href{mailto:Stefan.Zins@gesis.org}{Stefan.Zins@gesis.org}} and Matthias Sand\thanks{\href{mailto:Matthias.Sand@gesis.org}{Matthias.Sand@gesis.org}}}
\date{\today} 

\makebeamertitle

%Picture for cluster sampling

\begin{frame}{Cluster Sampling}
\begin{itemize}
\item[]<1-> Sampling elementary units is often not feasible (e.g. persons or businesses). Maybe there is no uniform sampling frame available to select them from, or it would be costly to do, because the selected elements would scatter to much over the a certain area and travel costs of interviewers would be to high. 

\item[]<2-> Thus, it is very common to select clusters, so called \emph{primary sampling units} (PSU's) that are populated by \emph{secondary sampling units} (SSU's).

\item[]<3-> Cluster sampling makes it still possible to obtain unbiased estimates but it can have a big influence on the variance. 
\end{itemize}

\end{frame}


\begin{frame}{Notation}
\begin{tabular}{rcp{8cm}}
$y_{ki}$      &=&value of the variable of interest for the $k$-th SSU in the $i$-th PSU \\
$N_{\RN{1}}$     &=&number of PSU's in the population \\
$N_i$       &=&number of SSU's in the $i$-th PSU \\
$N$         &=&total number of SSU's in the Population \\
$\mathcal{U}$  &=& set of SSU's in the population \\
$\mathcal{U}_{\RN{1}} $  &=& set of PSU's in the population \\
$\mathcal{U}_i$  &=& set of SSU's in the $i$-th PSU \\
$n_{\RN{1}}$         &=&number of PSU's in the sample \\
$n_i$   &=&number of SSU's in the sample from the $i$-th PSU\\
$\mathfrc{s}_{\RN{1}}$  &=& sample of PSU's \\
$\mathfrc{s}_i$         &=& sample SSU's from the $i$-th PSU \\ 
$p_{\RN{1}}(.)$         &=& sampling design of the PSU's \\
\end{tabular}

\end{frame}

\begin{frame}{Estimation in Case of $p_{\RN{1}}(.)=$ SRS}
All SSU's in the sampled PSU's are surveyed, thus
$$\tau_i=\sum_{k \in \mathcal{U}_i} y_{ki}$$
is known of all selected PSU's.
An unbiased estimator for the population mean is
$$\overline{y}_{\text{SRCS}} = \dfrac{N_{\RN{1}}}{N}  \sum_{i \in  \mathfrc{s}_{\RN{1}}}   \dfrac{\tau_i}{n_{\RN{1}}} $$
\onslide*<1>{with variance
$$\V{\overline{y}}_{\text{SRCS}} = \dfrac{N_{\RN{1}}^2}{N^2}  \left( 1 - \dfrac{n_{\RN{1}}}{N_{\RN{1}}} \right) \dfrac{V^2_{\tau}}{n_{\RN{1}}}\;,$$
where
 $V^2_{\tau} = \dfrac{1}{N_{\RN{1}} - 1} \sum_{i \in \mathcal{U}_{\RN{1}}} \left(\tau_i - \mu_\tau \right)^2$
and
 $\mu_\tau = \sum_{i \in \mathcal{U}_{\RN{1}}} \dfrac{\tau_i}{N_{\RN{1}}}$.
 }
\onslide*<2>{An unbiased variance estimator is
$$\Vest{\overline{y}_{\text{SRCS}}}_{\text{SRS}} = \dfrac{N_{\RN{1}}^2}{N^2} \left( 1 - \dfrac{n_{\RN{1}}}{N_{\RN{1}}} \right) \dfrac{s^2_{\tau}}{n_{\RN{1}}}\;,$$
where 
 $$
 s^2_{\tau} = \dfrac{1}{n_{\RN{1}}-1} \sum_{i \in \mathfrc{s}_{\RN{1}} } \left( \tau_i - \overline{\tau}  \right)^2\;.
 $$
with $\overline{\tau}=\sum_{i \in \mathfrc{s}_{\RN{1}} } \frac{\tau_i}{ n_{\RN{1}} }$. 
}
\end{frame}

\begin{frame}
\frametitle{Simple Cluster Sampling}
\onslide<1->{
