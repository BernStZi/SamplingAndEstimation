\documentclass[11pt,german,hideothersubsections]{beamer}\usepackage[]{graphicx}\usepackage[]{color}
%% maxwidth is the original width if it is less than linewidth
%% otherwise use linewidth (to make sure the graphics do not exceed the margin)
\makeatletter
\def\maxwidth{ %
  \ifdim\Gin@nat@width>\linewidth
    \linewidth
  \else
    \Gin@nat@width
  \fi
}
\makeatother

\definecolor{fgcolor}{rgb}{0.345, 0.345, 0.345}
\newcommand{\hlnum}[1]{\textcolor[rgb]{0.686,0.059,0.569}{#1}}%
\newcommand{\hlstr}[1]{\textcolor[rgb]{0.192,0.494,0.8}{#1}}%
\newcommand{\hlcom}[1]{\textcolor[rgb]{0.678,0.584,0.686}{\textit{#1}}}%
\newcommand{\hlopt}[1]{\textcolor[rgb]{0,0,0}{#1}}%
\newcommand{\hlstd}[1]{\textcolor[rgb]{0.345,0.345,0.345}{#1}}%
\newcommand{\hlkwa}[1]{\textcolor[rgb]{0.161,0.373,0.58}{\textbf{#1}}}%
\newcommand{\hlkwb}[1]{\textcolor[rgb]{0.69,0.353,0.396}{#1}}%
\newcommand{\hlkwc}[1]{\textcolor[rgb]{0.333,0.667,0.333}{#1}}%
\newcommand{\hlkwd}[1]{\textcolor[rgb]{0.737,0.353,0.396}{\textbf{#1}}}%

\usepackage{framed}
\makeatletter
\newenvironment{kframe}{%
 \def\at@end@of@kframe{}%
 \ifinner\ifhmode%
  \def\at@end@of@kframe{\end{minipage}}%
  \begin{minipage}{\columnwidth}%
 \fi\fi%
 \def\FrameCommand##1{\hskip\@totalleftmargin \hskip-\fboxsep
 \colorbox{shadecolor}{##1}\hskip-\fboxsep
     % There is no \\@totalrightmargin, so:
     \hskip-\linewidth \hskip-\@totalleftmargin \hskip\columnwidth}%
 \MakeFramed {\advance\hsize-\width
   \@totalleftmargin\z@ \linewidth\hsize
   \@setminipage}}%
 {\par\unskip\endMakeFramed%
 \at@end@of@kframe}
\makeatother

\definecolor{shadecolor}{rgb}{.97, .97, .97}
\definecolor{messagecolor}{rgb}{0, 0, 0}
\definecolor{warningcolor}{rgb}{1, 0, 1}
\definecolor{errorcolor}{rgb}{1, 0, 0}
\newenvironment{knitrout}{}{} % an empty environment to be redefined in TeX

\usepackage{alltt}

\usepackage{hyperref}
\usepackage{amsmath,nicefrac,booktabs,mathabx}
\usepackage{natbib}
\usepackage{url}
\usepackage{textpos}
\usepackage{listings}
\definecolor{Rblau}{rgb}{.3,.6,.9}

\lstset{language=R,
basicstyle=\ttfamily\footnotesize,
keywordstyle=\color{blue}\bfseries,
identifierstyle=\color{Rblau},
commentstyle=\color{gray},
stringstyle=\color{green}\ttfamily,
showstringspaces=false,
frame=tb}



\bibpunct{(}{)}{;}{a}{,}{,}
\usepackage[english]{babel}
\usepackage[latin1]{inputenc}
\usepackage{helvet}
\usepackage{graphicx}
\usepackage{color}
\usepackage{multirow,dcolumn}
\usepackage{ragged2e}
\usepackage{xcolor}
\usepackage{colortbl}
\usepackage{tikz}
\usetikzlibrary{calc}
\usepackage{booktabs}
\colorlet{tablesubheadcolor}{gray!25}
\colorlet{tableheadcolor}{gray!40}
\colorlet{tablerowcolor}{gray!15.0}
\usetheme[english]{Gesis}
\setbeamertemplate{navigation symbols}{}
\setbeamertemplate{footline}[frame number]%{\hspace*{.2cm}\insertframenumber}
\setbeamerfont{caption}{size=\footnotesize}
\usefonttheme[onlylarge]{structuresmallcapsserif} % alte Schrift

\newcommand{\R}[1]{{\tt \color{blue}  #1}}
  \newtheorem{thm}{Theorem}
  \newtheorem{rem}{Bemerkung}
  \newtheorem{lem}{Lemma}
  
  \definecolor{hellgrau}{rgb}   {0.109375,  0.40625,   0.51953125}
  \definecolor{dunkelgrau}{rgb} {0.009375,  0.30625,   0.41953125}
  \definecolor{dunkelgrau2}{rgb}{0.009375,  0.20625,   0.31953125}
  \definecolor{hellbraun}{rgb}  {0.9140625, 0.8984375, 0.8046875}
  \definecolor{hellbraun2}{rgb} {.95,       0.9,       0.8}
  \definecolor{alertred}{rgb}   {0.8515625, 0.3828125, 0.08984375}
  \definecolor{orange}{rgb}{1,0.5,0}
  
  
  \setbeamercolor{firstsecslide}{fg=white,bg=dunkelgrau}
  \setbeamertemplate{blocks}[rounded][shadow=true]
  
  \newcolumntype{d}[0]{D{,}{.}{6}}
  
  \newenvironment{itemizeol}{\begin{itemize}[<+->]}{\end{itemize}}
  \newenvironment{descriptionol}{\begin{description}[<+->]}{\end{description}}
  
  \newcolumntype{V}[1]{ {\RaggedRight\hspace{0pt}}p{#1}}

\newcommand{\emphred}[1]{\textcolor{alertred}{#1}}
\newcommand{\emphcol}[1]{\textcolor{dunkelgrau}{\slshape #1}}
 
 \setcounter{tocdepth}{1}
 \setbeamercolor*{section in toc}{fg=hellgrau}
\setbeamertemplate{bibliography item}[default]
 \makeatother
\addtobeamertemplate{frametitle}{}{%
 \begin{textblock*}{100mm}(.91\textwidth,-1cm)
 \includegraphics[height=1cm,width=2cm]{../common/pics/GESIS_Logo_kompakt_en.jpg}
 \end{textblock*}}
 \title[Day 3]{Sampling, Weighting and Estimation\\ \Large{Exercise 3} }
 %\subtitle{Umgang am Beispiel von Telefonstichproben}
 
 \author[M. Sand]{Stefan Zins, Matthias Sand\\ and Jan-Philipp Kolb\\ \vspace{.5cm} \footnotesize{GESIS - Leibniz Institute\\ for the Social Sciences}}
 %\institute{\includegraphics[width=4.5cm]{GESIS_Logo_informell}}
%  \date[]{\color{dunkelgrau}\footnotesize%
% \begin{minipage}{8cm}%
% \begin{center}%
%  \scriptsize{
% \textbf{GESIS Summer School}\\ \tiny{Cologne, Germany}%
%  }\\
% \vspace{0.25cm}
%  \textbf{August 24th, 2015}%
%  
%  \end{center}%
%  \end{minipage}}%
\IfFileExists{upquote.sty}{\usepackage{upquote}}{}
\begin{document}




\maketitle



%%%%%%%%%%%%%%%%%%%%%%%%%%%%%%%%%%%%%%%%%%%%%%%%%%
\begin{frame}[fragile]{Exercise 3a}

\begin{enumerate}

\item Download the data set for Germany of the 5th ESS-Round (Country File and Sampling Data)
\item[] \url{http://www.europeansocialsurvey.org/data/country.html?c=germany}
\item Estimate the design effect using the variables \alert{dweight},\alert{PSU} and \alert{agea} (model based approach) 
\item[] \alert{Advice:} The variable \alert{PSU} has to be a factor
\item Calculate the effective sample size
 

\end{enumerate}


\end{frame}
%%%%%%%%%%%%%%%%%%%%%%%%%%%%%%%%%%%%%%%%%%%%%%%%%%
\begin{frame}[fragile]{Design Effects: Model Based Approach}
\footnotesize{
\begin{block}{Model based approach}
\begin{equation*}
\hat{deff}=\hat{deff_p} * \hat{deff_c} = n\frac{\sum_{h=1}^ld_h^2 n_h}{(\sum_{h=1}^ld_h n_h)^2}*(1+(b^{*}-1)\rho)
\end{equation*}
\begin{equation*}
\hat{\rho}^{AOV}=\frac{MSB-MSW}{MSB+(K-1)MSW}
\end{equation*}
\begin{equation*}
MSB=\frac{SSB}{l-1}\text{;~~~~}MSW=\frac{SSW}{n-l}\text{;~~~~}K=\frac{1}{l-1}(n-\sum_{h=1}^l\frac{n_h^2}{n})\text{;}
\end{equation*}
\begin{equation*}
b^{*}=\frac{\sum_{l=1}^{L}(\sum_{i=1}^{n_h}w_{li})^2}{\sum_{l=1}^{L}\sum_{i=1}^{n_h}w_{li}^2}
\end{equation*}
\end{block}

\begin{itemize}
\item[] \alert{$n_h$} is the number of units per cluster; \alert{$b^{*}$} is the average cluster size; \alert{$\rho$} reflects the Intraclass Correlation Coefficient (ICC)
\item[$\Rightarrow$] $deff_p$ captures the design effect due to unequal inclusion probabilities
\end{itemize}
}
\end{frame}
%%%%%%%%%%%%%%%%%%%%%%%%%%%%%%%%%%%%%%%%%%%%%%%%%%%%
\begin{frame}[fragile]{Design Effects: Model Based Approach}
\begin{center}
\textbf{Obtaining \emph{MSB}, \emph{MSW} and \emph{$b^{*}$}:}
\end{center}

\footnotesize{
\begin{knitrout}
\definecolor{shadecolor}{rgb}{0.969, 0.969, 0.969}\color{fgcolor}\begin{kframe}
\begin{alltt}
\hlstd{Ger.d} \hlkwb{<-} \hlkwd{read.spss}\hlstd{(}\hlstr{"ESS5DE.spss/ESS5DE.sav"}\hlstd{,}
                   \hlkwc{to.data.frame} \hlstd{=} \hlnum{TRUE}\hlstd{,}
                   \hlkwc{use.value.labels} \hlstd{=} \hlnum{TRUE}\hlstd{)}
\hlstd{Ger.ctry} \hlkwb{<-} \hlkwd{read.spss}\hlstd{(}\hlstr{"ESS5_DE_SDDF.spss/ESS5_DE_SDDF.por"}\hlstd{,}
                      \hlkwc{to.data.frame} \hlstd{=} \hlnum{TRUE}\hlstd{,}
                      \hlkwc{use.value.labels} \hlstd{=} \hlnum{TRUE}\hlstd{)}

\hlkwd{colnames}\hlstd{(Ger.d)[}\hlnum{5}\hlstd{]} \hlkwb{<-} \hlstr{"IDNO"}
\hlstd{Ger} \hlkwb{<-} \hlkwd{merge}\hlstd{(Ger.d,Ger.ctry,}\hlkwc{by}\hlstd{=}\hlstr{"IDNO"}\hlstd{,} \hlkwc{all.x} \hlstd{=} \hlnum{TRUE}\hlstd{)}
\hlstd{Ger}\hlopt{$}\hlstd{PSU} \hlkwb{<-} \hlkwd{as.factor}\hlstd{(Ger}\hlopt{$}\hlstd{PSU)}
\hlstd{n} \hlkwb{<-} \hlkwd{nrow}\hlstd{(Ger)}
\hlstd{L} \hlkwb{<-} \hlkwd{length}\hlstd{(}\hlkwd{unique}\hlstd{(Ger}\hlopt{$}\hlstd{PSU))}
\end{alltt}
\end{kframe}
\end{knitrout}
}

\end{frame}
%%%%%%%%%%%%%%%%%%%%%%%%%%%%%%%%%%%%%%%%%%%%%%%%%%%%
\begin{frame}[fragile]{Design Effects: Model Based Approach}
\begin{center}
\textbf{Obtaining \emph{MSB}, \emph{MSW} and \emph{$b^{*}$}:}
\end{center}

\footnotesize{
\begin{knitrout}
\definecolor{shadecolor}{rgb}{0.969, 0.969, 0.969}\color{fgcolor}\begin{kframe}
\begin{alltt}
\hlcom{## deffc}
\hlstd{b} \hlkwb{<-} \hlkwd{sum}\hlstd{(}\hlkwd{tapply}\hlstd{(Ger}\hlopt{$}\hlstd{dweight,Ger}\hlopt{$}\hlstd{PSU,}
                \hlkwa{function}\hlstd{(}\hlkwc{x}\hlstd{)}\hlkwd{sum}\hlstd{(x)}\hlopt{^}\hlnum{2}\hlstd{))}\hlopt{/}\hlkwd{sum}\hlstd{(Ger}\hlopt{$}\hlstd{dweight}\hlopt{^}\hlnum{2}\hlstd{)}
\hlcom{# Calculate an anova for the regression model Age by PSU }
\hlcom{# (Coule also be any other Variable)}
\hlstd{lin.mod} \hlkwb{<-} \hlkwd{lm}\hlstd{(}\hlkwd{as.numeric}\hlstd{(Ger}\hlopt{$}\hlstd{agea)}\hlopt{~}\hlstd{Ger}\hlopt{$}\hlstd{PSU)}
\hlstd{SS} \hlkwb{<-} \hlkwd{anova}\hlstd{(lin.mod)}
\hlcom{#  MSB and MSW are the means of SSB and SSW}
\hlstd{MSB} \hlkwb{<-} \hlstd{SS}\hlopt{$}\hlstd{`Mean Sq`[}\hlnum{1}\hlstd{]}
\hlstd{MSW} \hlkwb{<-} \hlstd{SS}\hlopt{$}\hlstd{`Mean Sq`[}\hlnum{2}\hlstd{]}
\end{alltt}
\end{kframe}
\end{knitrout}

}

\end{frame}

%%%%%%%%%%%%%%%%%%%%%%%%%%%%%%%%%%%%%%%%%%%%%%%%%%%+
\begin{frame}[fragile]{Exercise 3b}

\begin{enumerate}
\item Download the following R-Script: \url{https://github.com/BernStZi/SamplingAndEstimation/blob/short/tutorial/Samples_for_EX3b.R} to generate a Multistage- and a Cluster- Sample for the belgianmunicipalities data set
\item Calculate the mean income of the population
\item Estimate the mean income from both samples, using the \R{survey} package and compare the results



\end{enumerate}


\end{frame}
%%%%%%%%%%%%%%%%%%%%%%%%%%%%%%%%%%%%%%%%%%%%%%%%%%
\begin{frame}[fragile]{Multistage- and Cluster-Samples \\ with the \R{survey} package}

\begin{knitrout}
\definecolor{shadecolor}{rgb}{0.969, 0.969, 0.969}\color{fgcolor}\begin{kframe}
\begin{alltt}
\hlstd{surv} \hlkwb{<-} \hlkwd{svydesign}\hlstd{(}\hlkwc{id}\hlstd{=}\hlopt{~}\hlstd{Commune}\hlopt{+}\hlstd{id,}\hlkwc{fpc}\hlstd{=}\hlopt{~}\hlstd{prob1}\hlopt{+}\hlstd{prob2,}
                  \hlkwc{data}\hlstd{=Data.be,}\hlkwc{pps}\hlstd{=}\hlstr{"brewer"}\hlstd{)}
\end{alltt}
\end{kframe}
\end{knitrout}
\footnotesize{
\begin{itemize}
\item In \emph{Exercise 1} we had a single-stage sample, therefore the argument \R{id} has been set to 0 or 1
\item[$\Rightarrow$] In case of a multi-stage sampling approach, every sampling stage has to be defined
\begin{itemize}
\item[$\Rightarrow$] PSU: \emph{Commune}; SSU: \emph{id}
\end{itemize}
\item This also applies for the \R{fpc}-argument 
\item[$\Rightarrow$] \emph{prob1} reflects the porbability of inclusion for each PSU in the sample and \emph{prob2} the probability of inclusion for each SSU
\item[] \alert{Note:} altough $prob1*prob2=n/N$ in this sample, it cannot be treated like a SRS

\end{itemize}
}


\end{frame}
%%%%%%%%%%%%%%%%%%%%%%%%%%%%%%%%%%%%%%%%%%%%%%%%%%
\begin{frame}[fragile]{Multistage- and Cluster-Samples \\ with the \R{survey} package}

\begin{knitrout}
\definecolor{shadecolor}{rgb}{0.969, 0.969, 0.969}\color{fgcolor}\begin{kframe}
\begin{alltt}
\hlstd{surv} \hlkwb{<-} \hlkwd{svydesign}\hlstd{(}\hlkwc{id}\hlstd{=}\hlopt{~}\hlstd{Commune}\hlopt{+}\hlstd{id,}\hlkwc{fpc}\hlstd{=}\hlopt{~}\hlstd{prob1}\hlopt{+}\hlstd{prob2,}
                  \hlkwc{data}\hlstd{=Data.be,} \hlkwc{pps}\hlstd{=}\hlstr{"brewer"}\hlstd{)}
\end{alltt}
\end{kframe}
\end{knitrout}
\begin{itemize}
\item \R{pps} should be used to define the design information; usually the second order probability of inclusion
\item[$\Rightarrow$] If the second order probability of inclusion are unknown (or too complex to calculate), a brewer approximation can be applied to estimate the joint inclusion probabilities
\end{itemize}

\end{frame}

\end{document}
