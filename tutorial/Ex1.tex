\documentclass[]{article}
\usepackage{lmodern}
\usepackage{amssymb,amsmath}
\usepackage{ifxetex,ifluatex}
\usepackage{fixltx2e} % provides \textsubscript
\ifnum 0\ifxetex 1\fi\ifluatex 1\fi=0 % if pdftex
  \usepackage[T1]{fontenc}
  \usepackage[utf8]{inputenc}
\else % if luatex or xelatex
  \ifxetex
    \usepackage{mathspec}
    \usepackage{xltxtra,xunicode}
  \else
    \usepackage{fontspec}
  \fi
  \defaultfontfeatures{Mapping=tex-text,Scale=MatchLowercase}
  \newcommand{\euro}{€}
\fi
% use upquote if available, for straight quotes in verbatim environments
\IfFileExists{upquote.sty}{\usepackage{upquote}}{}
% use microtype if available
\IfFileExists{microtype.sty}{%
\usepackage{microtype}
\UseMicrotypeSet[protrusion]{basicmath} % disable protrusion for tt fonts
}{}
\usepackage[margin=1in]{geometry}
\usepackage{color}
\usepackage{fancyvrb}
\newcommand{\VerbBar}{|}
\newcommand{\VERB}{\Verb[commandchars=\\\{\}]}
\DefineVerbatimEnvironment{Highlighting}{Verbatim}{commandchars=\\\{\}}
% Add ',fontsize=\small' for more characters per line
\usepackage{framed}
\definecolor{shadecolor}{RGB}{248,248,248}
\newenvironment{Shaded}{\begin{snugshade}}{\end{snugshade}}
\newcommand{\KeywordTok}[1]{\textcolor[rgb]{0.13,0.29,0.53}{\textbf{{#1}}}}
\newcommand{\DataTypeTok}[1]{\textcolor[rgb]{0.13,0.29,0.53}{{#1}}}
\newcommand{\DecValTok}[1]{\textcolor[rgb]{0.00,0.00,0.81}{{#1}}}
\newcommand{\BaseNTok}[1]{\textcolor[rgb]{0.00,0.00,0.81}{{#1}}}
\newcommand{\FloatTok}[1]{\textcolor[rgb]{0.00,0.00,0.81}{{#1}}}
\newcommand{\CharTok}[1]{\textcolor[rgb]{0.31,0.60,0.02}{{#1}}}
\newcommand{\StringTok}[1]{\textcolor[rgb]{0.31,0.60,0.02}{{#1}}}
\newcommand{\CommentTok}[1]{\textcolor[rgb]{0.56,0.35,0.01}{\textit{{#1}}}}
\newcommand{\OtherTok}[1]{\textcolor[rgb]{0.56,0.35,0.01}{{#1}}}
\newcommand{\AlertTok}[1]{\textcolor[rgb]{0.94,0.16,0.16}{{#1}}}
\newcommand{\FunctionTok}[1]{\textcolor[rgb]{0.00,0.00,0.00}{{#1}}}
\newcommand{\RegionMarkerTok}[1]{{#1}}
\newcommand{\ErrorTok}[1]{\textbf{{#1}}}
\newcommand{\NormalTok}[1]{{#1}}
\usepackage{longtable,booktabs}
\ifxetex
  \usepackage[setpagesize=false, % page size defined by xetex
              unicode=false, % unicode breaks when used with xetex
              xetex]{hyperref}
\else
  \usepackage[unicode=true]{hyperref}
\fi
\hypersetup{breaklinks=true,
            bookmarks=true,
            pdfauthor={Stefan Zins, Matthias Sand and Jan-Philipp Kolb},
            pdftitle={Introduction to the Analysis of Sample Surveys with R - Exercise 1},
            colorlinks=true,
            citecolor=blue,
            urlcolor=blue,
            linkcolor=magenta,
            pdfborder={0 0 0}}
\urlstyle{same}  % don't use monospace font for urls
\setlength{\parindent}{0pt}
\setlength{\parskip}{6pt plus 2pt minus 1pt}
\setlength{\emergencystretch}{3em}  % prevent overfull lines
\setcounter{secnumdepth}{0}

%%% Use protect on footnotes to avoid problems with footnotes in titles
\let\rmarkdownfootnote\footnote%
\def\footnote{\protect\rmarkdownfootnote}

%%% Change title format to be more compact
\usepackage{titling}

% Create subtitle command for use in maketitle
\newcommand{\subtitle}[1]{
  \posttitle{
    \begin{center}\large#1\end{center}
    }
}

\setlength{\droptitle}{-2em}
  \title{Introduction to the Analysis of Sample Surveys with R - Exercise 1}
  \pretitle{\vspace{\droptitle}\centering\huge}
  \posttitle{\par}
  \author{Stefan Zins, Matthias Sand and Jan-Philipp Kolb}
  \preauthor{\centering\large\emph}
  \postauthor{\par}
  \predate{\centering\large\emph}
  \postdate{\par}
  \date{4 February 2016}



\begin{document}

\maketitle


\begin{center}\rule{0.5\linewidth}{\linethickness}\end{center}

\begin{enumerate}
\def\labelenumi{\arabic{enumi}.}
\itemsep1pt\parskip0pt\parsep0pt
\item
  Download the ESS dataset for
  \href{http://www.europeansocialsurvey.org/data/country.html?c=sweden}{Sweden}
  (Survey Data and Sampling Design Data File (SDDF)) of the 5th round
\item
  Setup your workspace and load the R-packages
  \href{https://cran.r-project.org/web/packages/foreign/foreign.pdf}{foreign}
  and
  \href{https://cran.r-project.org/web/packages/survey/index.html}{survey}
\item
  Load the ESS dataset and the SDDF
\item
  Merge both data frames by their ID-variable, using the
  \texttt{merge()} command
\end{enumerate}

\begin{center}\rule{0.5\linewidth}{\linethickness}\end{center}

\begin{enumerate}
\def\labelenumi{\arabic{enumi}.}
\setcounter{enumi}{4}
\itemsep1pt\parskip0pt\parsep0pt
\item
  Determine the sampling strategy (Inspect the variables
  \texttt{PSU},\texttt{STRATFY} and \texttt{PROB})
\item
  Add the variable \texttt{N} for the population size to your data
  frame. \texttt{N} can be calulated by
  \[N= dweight* pweight *10000*n \text{,}\] where \(n\) refers to the
  sample size
\item
  Create a \texttt{svydesign} object from the dataset for Sweden using
  the \texttt{survey} package
\item
  Estimate the total and mean of the variable \texttt{tvtot}
\end{enumerate}

\begin{center}\rule{0.5\linewidth}{\linethickness}\end{center}

\section{The survey package}\label{the-survey-package}

\begin{itemize}
\itemsep1pt\parskip0pt\parsep0pt
\item
  The survey package provides a large range of applications for complex
  survey samples
\item
  Typically, the first step is to define a survey object with the
  \texttt{svydesign()} command
\end{itemize}

\subsubsection{Simple Survey Object (Simple Random
Sample)}\label{simple-survey-object-simple-random-sample}

\begin{Shaded}
\begin{Highlighting}[]
\KeywordTok{data}\NormalTok{(api)}

\NormalTok{surv.obj <-}\StringTok{ }\KeywordTok{svydesign}\NormalTok{(}\DataTypeTok{id=}\NormalTok{~}\DecValTok{1}\NormalTok{,}\DataTypeTok{fpc =} \NormalTok{~fpc, }\DataTypeTok{data =} \NormalTok{apisrs)}
\end{Highlighting}
\end{Shaded}

\begin{itemize}
\item
  \texttt{id} specifies the identifier of PSU and SSU;\texttt{id} \(=\)
  \textasciitilde{} 0 or \(=\) \textasciitilde{} 1 stipulates a single
  stage sampling
\item
  For multi-stage samples the \texttt{id} argument should always specify
  a formula with the (cluster-) identifier at each stage
\item
  \texttt{fpc} should be used for the finite population correction

  \(\Rightarrow\) Either as the total population size of each stratum or
  as a fraction of the total population that has been sampled\\
\item
  \texttt{data} reflects the data set for which the design object should
  be defined
\end{itemize}

\begin{center}\rule{0.5\linewidth}{\linethickness}\end{center}

\begin{longtable}[c]{@{}ll@{}}
\toprule
** ** & \textbf{Important Commands}\tabularnewline
\midrule
\endhead
\texttt{svytotal} & returns the estimated total of a variable and its
standard error (\(+ deff\))\tabularnewline
\texttt{svymean} & returns the estimated mean of a variable and its
standard error (\(+ deff\))\tabularnewline
\texttt{svyquantile} & Computes quantiles for data from complex
surveys\tabularnewline
\texttt{svyvar} & Computes variances for data from complex
surveys\tabularnewline
\texttt{weights} & Returns the (design) weights of a survey
object\tabularnewline
\texttt{calibrate} & Calibration of a data set (uses the GREG-Estimator
by default)\tabularnewline
\bottomrule
\end{longtable}

\begin{Shaded}
\begin{Highlighting}[]
\KeywordTok{svytotal}\NormalTok{(~api00,surv.obj)}
\end{Highlighting}
\end{Shaded}

\begin{verbatim}
##         total    SE
## api00 4066888 57293
\end{verbatim}

\end{document}
