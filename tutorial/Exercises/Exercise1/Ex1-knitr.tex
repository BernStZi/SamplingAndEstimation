\documentclass[11pt,german,hideothersubsections]{beamer}

\usepackage{hyperref}
\usepackage{amsmath,nicefrac,booktabs,mathabx}
\usepackage{natbib}
\usepackage{url}
\usepackage{textpos}
\usepackage{listings}
\definecolor{Rblau}{rgb}{.3,.6,.9}

\lstset{language=R,
basicstyle=\ttfamily\footnotesize,
keywordstyle=\color{blue}\bfseries,
identifierstyle=\color{Rblau},
commentstyle=\color{gray},
stringstyle=\color{green}\ttfamily,
showstringspaces=false,
frame=tb}



\bibpunct{(}{)}{;}{a}{,}{,}
\usepackage[english]{babel}
\usepackage[latin1]{inputenc}
\usepackage{helvet}
\usepackage{graphicx}
\usepackage{color}
\usepackage{multirow,dcolumn}
\usepackage{ragged2e}
\usepackage{xcolor}
\usepackage{colortbl}
\usepackage{tikz}
\usetikzlibrary{calc}
\usepackage{booktabs}
\colorlet{tablesubheadcolor}{gray!25}
\colorlet{tableheadcolor}{gray!40}
\colorlet{tablerowcolor}{gray!15.0}
\usetheme[english]{Gesis}
\setbeamertemplate{navigation symbols}{}
\setbeamertemplate{footline}[frame number]%{\hspace*{.2cm}\insertframenumber}
\setbeamerfont{caption}{size=\footnotesize}
\usefonttheme[onlylarge]{structuresmallcapsserif} % alte Schrift

\newcommand{\R}[1]{{\tt \color{blue}  #1}}
  \newtheorem{thm}{Theorem}
  \newtheorem{rem}{Bemerkung}
  \newtheorem{lem}{Lemma}
  
  \definecolor{hellgrau}{rgb}   {0.109375,  0.40625,   0.51953125}
  \definecolor{dunkelgrau}{rgb} {0.009375,  0.30625,   0.41953125}
  \definecolor{dunkelgrau2}{rgb}{0.009375,  0.20625,   0.31953125}
  \definecolor{hellbraun}{rgb}  {0.9140625, 0.8984375, 0.8046875}
  \definecolor{hellbraun2}{rgb} {.95,       0.9,       0.8}
  \definecolor{alertred}{rgb}   {0.8515625, 0.3828125, 0.08984375}
  \definecolor{orange}{rgb}{1,0.5,0}
  
  
  \setbeamercolor{firstsecslide}{fg=white,bg=dunkelgrau}
  \setbeamertemplate{blocks}[rounded][shadow=true]
  
  \newcolumntype{d}[0]{D{,}{.}{6}}
  
  \newenvironment{itemizeol}{\begin{itemize}[<+->]}{\end{itemize}}
  \newenvironment{descriptionol}{\begin{description}[<+->]}{\end{description}}
  
  \newcolumntype{V}[1]{ {\RaggedRight\hspace{0pt}}p{#1}}

\newcommand{\emphred}[1]{\textcolor{alertred}{#1}}
\newcommand{\emphcol}[1]{\textcolor{dunkelgrau}{\slshape #1}}
 
 \setcounter{tocdepth}{1}
 \setbeamercolor*{section in toc}{fg=hellgrau}
\setbeamertemplate{bibliography item}[default]
 \makeatother
\addtobeamertemplate{frametitle}{}{%
 \begin{textblock*}{100mm}(.91\textwidth,-1cm)
 \includegraphics[height=1cm,width=2cm]{../../../lecture/graphs/logos/GESIS_Logo_kompakt_en.jpg}
 \end{textblock*}}
 \title[Day 1]{Sampling, Weighting and Estimation\\ \Large{Exercise 1} }
 %\subtitle{Umgang am Beispiel von Telefonstichproben}
 
 \author[M. Sand]{Stefan Zins, Matthias Sand\\ and Jan-Philipp Kolb\\ \vspace{.5cm} \footnotesize{GESIS - Leibniz Institute\\ for the Social Sciences}}
 %\institute{\includegraphics[width=4.5cm]{GESIS_Logo_informell}}
%  \date[]{\color{dunkelgrau}\footnotesize%
% \begin{minipage}{8cm}%
% \begin{center}%
%  \scriptsize{
% \textbf{GESIS Summer School}\\ \tiny{Cologne, Germany}%
%  }\\
% \vspace{0.25cm}
%  \textbf{August 24th, 2015}%
%  
%  \end{center}%
%  \end{minipage}}%

\usepackage{Sweave}
\begin{document}
\Sconcordance{concordance:Ex1-knitr.tex:Ex1-knitr.Rnw:%
1 98 1 49 0 1 10 3 1 1 7 41 1 1 4 7 1 3 0 9 1 4 0 30 1 8 0 7 1}


\begin{Schunk}
\begin{Sinput}
> library(knitr)
> opts_chunk$set(
+ concordance=TRUE
+ )
\end{Sinput}
\end{Schunk}

\maketitle


%%%%%%%%%%%%%%%%%%%%%%%%%%%%%%%%%%%%%%%%%%%%%%%%%%
\begin{frame}[fragile]{Exercise 1}

\begin{enumerate}

\item Download the ESS dataset for Sweden (Sampling Data and Country File) of the 5th round
\item[] \alert{\url{http://www.europeansocialsurvey.org/data/country.html?c=sweden}}
\item Setup your workspace and load the R-packages \R{foreign} and \R{survey}
\item Load the ESS dataset and the country file
\item Merge both data frames by their ID-variable, using the \R{merge()}-command

\end{enumerate}

\end{frame}
%%%%%%%%%%%%%%%%%%%%%%%%%%%%%%%%%%%%%%%%%%%%%%%%%%%%%
\begin{frame}[fragile]{Exercise 1}

\begin{enumerate}

\setcounter{enumi}{4}
\item Determine the sampling strategy (Inspect the variables \alert{SAMPPOIN},\alert{STRATFY} and \alert{PROB})
\item Add the variable \R{N}$=\R{dweight}*\R{pweight}*10000*n$ to your data frame that equals the total population size for the target population of the ESS Round 5 of Sweden
\item Convert your data frame to a survey object (\R{survey}-Package)
\item Estimate the total and mean of the variable \R{tvtot}

\end{enumerate}

\end{frame}
%%%%%%%%%%%%%%%%%%%%%%%%%%%%%%%%%%%%%%%%%%%%%%%%%%%%%%
\begin{frame}[fragile]{The \R{survey}-Package}
\footnotesize{
\begin{itemize}
\item The survey package provides a large range of applications for complex survey samples
\item Typically, the first step is to define a survey object with the \R{svydesign()} command
\end{itemize}
\begin{center}
\textbf{Simple survey object (Simple Random Sample)}
\end{center}


\begin{Schunk}
\begin{Sinput}
 library(Matrix)
 library(survey)
\end{Sinput}
\end{Schunk}

\begin{Schunk}
\begin{Sinput}
 data(api)
 surv.obj <- svydesign(id=~1,fpc = ~fpc, data = apisrs)
\end{Sinput}
\end{Schunk}
}
\end{frame}
%%%%%%%%%%%%%%%%%%%%%%%%%%%%%%%%%%%%%%%%%%%%%%%%%%%%%%%%%
\begin{frame}[fragile]{The survey package}
%\frametitle{\vspace{-.05cm}\begin{center}\footnotesize{The survey package}\end{center}}
\footnotesize{
\begin{Schunk}
\begin{Sinput}
 surv.obj <- svydesign(id=~1,fpc = ~fpc, data = apisrs)
\end{Sinput}
\end{Schunk}
}
\begin{itemize}
\item \R{id} specifies the identifier of PSU and SSU;\R{id$=$ \textasciitilde0} or \R{id$=$\textasciitilde1} stipulates a single stage sampling
\item For multi-stage samples the \R{id} argument should always specify a formula with the cluster-identifier at each stage
\item \R{fpc} should be used for the finite population correction
\item[$\Rightarrow$] Either as the total population size of each stratum or as a fraction of the total population that has been sampled 
\item \R{data} reflects the data set for which the design object should be defined
\end{itemize}
\end{frame}
%%%%%%%%%%%%%%%%%%%%%%%%%%%%%%%%%%%%%%%%%%%%%%%%%%%%%%%%%
%%%%%%%%%%%%%%%%%%%%%%%%%%%%%%%%%%%%%%%%%%%%%%%%%%%%%%%%%
\begin{frame}[fragile]{The survey package}
%\frametitle{\vspace{-.05cm}\begin{center}\footnotesize{The survey package}\end{center}}
\footnotesize{
\begin{center}
\textbf{Important commands}
\end{center}
\begin{tabular}{ll}
\R{svytotal} & returns the estimated total of a variable \\ & and its standard error ($+ deff$)\\
\R{svymean} & returns the estimated mean of a variable\\ & and its standard error ($+ deff$)\\
\R{svyquantile} & Computes quantiles for data from complex surveys\\
\R{svyvar} & Computes variances  for data from complex surveys\\
\R{weights} & Returns the (design) weights of a survey object\\
\R{calibrate} & Calibration of a data set (uses the GREG-Estimator)\\
... & ...\\
\end{tabular}

\begin{Schunk}
\begin{Sinput}
 svytotal(~api00,surv.obj)
\end{Sinput}
\begin{Soutput}
        total    SE
api00 4066888 57293
\end{Soutput}
\end{Schunk}
}
\end{frame}




\end{document}
