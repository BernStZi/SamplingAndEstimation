\documentclass[11pt,german,hideothersubsections]{beamer}\usepackage[]{graphicx}\usepackage[]{color}
%% maxwidth is the original width if it is less than linewidth
%% otherwise use linewidth (to make sure the graphics do not exceed the margin)
\makeatletter
\def\maxwidth{ %
  \ifdim\Gin@nat@width>\linewidth
    \linewidth
  \else
    \Gin@nat@width
  \fi
}
\makeatother

\definecolor{fgcolor}{rgb}{0.345, 0.345, 0.345}
\newcommand{\hlnum}[1]{\textcolor[rgb]{0.686,0.059,0.569}{#1}}%
\newcommand{\hlstr}[1]{\textcolor[rgb]{0.192,0.494,0.8}{#1}}%
\newcommand{\hlcom}[1]{\textcolor[rgb]{0.678,0.584,0.686}{\textit{#1}}}%
\newcommand{\hlopt}[1]{\textcolor[rgb]{0,0,0}{#1}}%
\newcommand{\hlstd}[1]{\textcolor[rgb]{0.345,0.345,0.345}{#1}}%
\newcommand{\hlkwa}[1]{\textcolor[rgb]{0.161,0.373,0.58}{\textbf{#1}}}%
\newcommand{\hlkwb}[1]{\textcolor[rgb]{0.69,0.353,0.396}{#1}}%
\newcommand{\hlkwc}[1]{\textcolor[rgb]{0.333,0.667,0.333}{#1}}%
\newcommand{\hlkwd}[1]{\textcolor[rgb]{0.737,0.353,0.396}{\textbf{#1}}}%

\usepackage{framed}
\makeatletter
\newenvironment{kframe}{%
 \def\at@end@of@kframe{}%
 \ifinner\ifhmode%
  \def\at@end@of@kframe{\end{minipage}}%
  \begin{minipage}{\columnwidth}%
 \fi\fi%
 \def\FrameCommand##1{\hskip\@totalleftmargin \hskip-\fboxsep
 \colorbox{shadecolor}{##1}\hskip-\fboxsep
     % There is no \\@totalrightmargin, so:
     \hskip-\linewidth \hskip-\@totalleftmargin \hskip\columnwidth}%
 \MakeFramed {\advance\hsize-\width
   \@totalleftmargin\z@ \linewidth\hsize
   \@setminipage}}%
 {\par\unskip\endMakeFramed%
 \at@end@of@kframe}
\makeatother

\definecolor{shadecolor}{rgb}{.97, .97, .97}
\definecolor{messagecolor}{rgb}{0, 0, 0}
\definecolor{warningcolor}{rgb}{1, 0, 1}
\definecolor{errorcolor}{rgb}{1, 0, 0}
\newenvironment{knitrout}{}{} % an empty environment to be redefined in TeX

\usepackage{alltt}

\usepackage{hyperref}
\usepackage{amsmath,nicefrac,booktabs,mathabx}
\usepackage{natbib}
\usepackage{url}
\usepackage{textpos}
\usepackage{listings}
\definecolor{Rblau}{rgb}{.3,.6,.9}

\lstset{language=R,
basicstyle=\ttfamily\footnotesize,
keywordstyle=\color{blue}\bfseries,
identifierstyle=\color{Rblau},
commentstyle=\color{gray},
stringstyle=\color{green}\ttfamily,
showstringspaces=false,
frame=tb}



\bibpunct{(}{)}{;}{a}{,}{,}
\usepackage[english]{babel}
\usepackage[latin1]{inputenc}
\usepackage{helvet}
\usepackage{graphicx}
\usepackage{color}
\usepackage{multirow,dcolumn}
\usepackage{ragged2e}
\usepackage{xcolor}
\usepackage{colortbl}
\usepackage{tikz}
\usetikzlibrary{calc}
\usepackage{booktabs}
\colorlet{tablesubheadcolor}{gray!25}
\colorlet{tableheadcolor}{gray!40}
\colorlet{tablerowcolor}{gray!15.0}
\usetheme[english]{Gesis}
\setbeamertemplate{navigation symbols}{}
\setbeamertemplate{footline}[frame number]%{\hspace*{.2cm}\insertframenumber}
\setbeamerfont{caption}{size=\footnotesize}
\usefonttheme[onlylarge]{structuresmallcapsserif} % alte Schrift

\newcommand{\R}[1]{{\tt \color{blue}  #1}}
  \newtheorem{thm}{Theorem}
  \newtheorem{rem}{Bemerkung}
  \newtheorem{lem}{Lemma}
  
  \definecolor{hellgrau}{rgb}   {0.109375,  0.40625,   0.51953125}
  \definecolor{dunkelgrau}{rgb} {0.009375,  0.30625,   0.41953125}
  \definecolor{dunkelgrau2}{rgb}{0.009375,  0.20625,   0.31953125}
  \definecolor{hellbraun}{rgb}  {0.9140625, 0.8984375, 0.8046875}
  \definecolor{hellbraun2}{rgb} {.95,       0.9,       0.8}
  \definecolor{alertred}{rgb}   {0.8515625, 0.3828125, 0.08984375}
  \definecolor{orange}{rgb}{1,0.5,0}
  
  
  \setbeamercolor{firstsecslide}{fg=white,bg=dunkelgrau}
  \setbeamertemplate{blocks}[rounded][shadow=true]
  
  \newcolumntype{d}[0]{D{,}{.}{6}}
  
  \newenvironment{itemizeol}{\begin{itemize}[<+->]}{\end{itemize}}
  \newenvironment{descriptionol}{\begin{description}[<+->]}{\end{description}}
  
  \newcolumntype{V}[1]{ {\RaggedRight\hspace{0pt}}p{#1}}

\newcommand{\emphred}[1]{\textcolor{alertred}{#1}}
\newcommand{\emphcol}[1]{\textcolor{dunkelgrau}{\slshape #1}}
 
 \setcounter{tocdepth}{1}
 \setbeamercolor*{section in toc}{fg=hellgrau}
\setbeamertemplate{bibliography item}[default]
 \makeatother
\addtobeamertemplate{frametitle}{}{%
 \begin{textblock*}{100mm}(.91\textwidth,-1cm)
 \includegraphics[height=1cm,width=2cm]{../common/pics/GESIS_Logo_kompakt_en.jpg}
 \end{textblock*}}
 \title[Day 1]{Sampling, Weighting and Estimation\\ \Large{Exercise 2} }
 %\subtitle{Umgang am Beispiel von Telefonstichproben}
 
 \author[M. Sand]{Stefan Zins, Matthias Sand\\ and Jan-Philipp Kolb\\ \vspace{.5cm} \footnotesize{GESIS - Leibniz Institute\\ for the Social Sciences}}
 %\institute{\includegraphics[width=4.5cm]{GESIS_Logo_informell}}
%  \date[]{\color{dunkelgrau}\footnotesize%
% \begin{minipage}{8cm}%
% \begin{center}%
%  \scriptsize{
% \textbf{GESIS Summer School}\\ \tiny{Cologne, Germany}%
%  }\\
% \vspace{0.25cm}
%  \textbf{August 24th, 2015}%
%  
%  \end{center}%
%  \end{minipage}}%
\IfFileExists{upquote.sty}{\usepackage{upquote}}{}
\begin{document}



\maketitle

%%%%%%%%%%%%%%%%%%%%%%%%%%%%%%%%%%%%%%%%%%%%%%%%%%
\begin{frame}[fragile]{Exercise 2: Stratification and Allocation}

\begin{enumerate}
\item Load the \R{survey} package and the \R{api} datasets in it.
\item The dataset \R{apistrat} is a sample of schools from \R{apipop} stratified by \R{stype}. Assuming the selection within the strata was done by SRS, define a \R{svydesign} object that enables your to make unbiased point and variance estimates. Estimate the mean of \R{api00}.
\item Now you should try different allocations. Using \R{stype} again as a stratification variable calculate the allocation of a sample of 60 schools from \R{apipop}, using equal, proportional to the number of schools, and optimal with regard to \R{api99} allocation.
\item Select a StrSRS from \R{apipop} for each of your allocations.
\item Estimate again the mean of \R{api00} from your three different samples.

\end{enumerate}

\end{frame}


\begin{frame}[fragile]{A Function for Stratified Sampling}
\begin{knitrout}\footnotesize
\definecolor{shadecolor}{rgb}{0.969, 0.969, 0.969}\color{fgcolor}\begin{kframe}
\begin{alltt}
\hlcom{############################# Input ###############################}
\hlcom{#strind: the stratification varaible; a population length vector.}
\hlcom{#nh:     allocation; a vector with elements named after the strata.}
\hlcom{#replace: logical; sampling with or without replacement.}
\hlcom{############################# Output ##############################}
\hlcom{#A numeric vector containing the sampled }
\hlcom{#rows of the population dataset.}

\hlstd{strSR.sample} \hlkwb{<-} \hlkwa{function}\hlstd{(}\hlkwc{strind}\hlstd{,} \hlkwc{nh}\hlstd{,} \hlkwc{replace}\hlstd{=}\hlnum{FALSE}\hlstd{)\{}
   \hlstd{Nh}   \hlkwb{<-} \hlkwd{table}\hlstd{(strind)[}\hlkwd{names}\hlstd{(nh)]}
   \hlstd{h.id} \hlkwb{<-} \hlkwd{split}\hlstd{(}\hlnum{1}\hlopt{:}\hlkwd{sum}\hlstd{(Nh), strind)[}\hlkwd{names}\hlstd{(nh)]}
   \hlstd{sam} \hlkwb{<-} \hlkwd{mapply}\hlstd{(} \hlkwa{function}\hlstd{(}\hlkwc{x}\hlstd{,}\hlkwc{y}\hlstd{)} \hlkwd{sample}\hlstd{(x, y,} \hlkwc{replace}\hlstd{=replace)}
                 \hlstd{, Nh, nh,} \hlkwc{SIMPLIFY} \hlstd{= F)}
   \hlkwd{unlist}\hlstd{(}\hlkwd{mapply}\hlstd{(}\hlkwa{function}\hlstd{(}\hlkwc{x}\hlstd{,}\hlkwc{y}\hlstd{) x[y]}
                 \hlstd{, h.id}
                 \hlstd{, sam,} \hlkwc{SIMPLIFY} \hlstd{= F)}
          \hlstd{,}\hlkwc{use.names} \hlstd{=} \hlnum{FALSE}\hlstd{)}

\hlstd{\}}
\end{alltt}
\end{kframe}
\end{knitrout}
\end{frame}

\end{document}
